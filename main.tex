\documentclass[12pt]{article}

\usepackage{handouts} % multi-handout handling
\usepackage[]{solution} % include [show] to make solutions visible.
\usepackage{physhandout} % physics-specific stuff

\usepackage{hyperref} % currently required, will make optional in future
\usepackage{graphicx}

\courseinfo{Course Name (Year)} % appears bottom-left footer
\author{} % does nothing unless you use [plain] option to handouts
\date{} % same

\copyrightinfo{\url{https://github.com/tdwiser/physhandout}} %sets a (global) bottom-right footer. Can be set inside a handout BUT carries over to following handouts
% \problemname{...} %changes the `Problem #' text to `... #' when you \begin{problem}...\end{problem}

\inithandout{lecture}{Lecture} % creates a new type of handout, a Lecture, which can be invoked with \begin{lecture}...\end{lecture}
% \renewcommand{\lectureheader}[2]{\Large Lecture #1: #2} % can be used to insert something at the top of each lecture (say)

\inithandout{activity}{Activity}
\inithandout{pset}{Problem Set}
% \inithandout{misc}{Handout} % can't call it ``handout'' without causing some numbering issues (or worse?)

\onlylecture{} % don't generate any of the lectures
\onlyactivity{sample_activity} % only produce sample_activity, nothing else
% \onlypset{} % if you leave out an \only, they will all be made

\begin{document}

\begin{activity}[lame_activity]{A Lame Activity}
	We don't want this activity show up in the pdf file, so it's not listed in the \verb|\onlyactivity{}| command above.
\end{activity}

\begin{activity}[sample_activity]{A Sample Activity}
	Here is the \LaTeX\ code for this sample activity, which is automatically numbered (this is Activity~\ref{activity:sample_activity}; note that Activity~\ref{activity:lame_activity} can still be \verb|\ref|'d even though it isn't in this file).
	
	You could also \verb|\input{file.tex}| here instead to keep things clean (recommended!)
	
	\subsection*{Solution macros:}
		Here is an equation: $E=mc^2$
\begin{solution}
	    This is the solution to that equation, in long form: $m=E/c^2$
\end{solution}

	    Here is an inline solution (\soln{answer}) which leaves space for the answer.
    
	    \bigsoln{This a multi
    
	    paragraph
    
	    solution.
	    } (leaves one par-ish of space)
   
	   	Here is an inline solution (\solnx{answer}) which doesn't leave space
\end{activity}

\begin{pset}[ps1]{A Problem Set}
	This Problem Set (which is number~\ref{pset:ps1}, numbered independently from the activities, etc.) will show up in the pdf file because there is no \verb|\onlypset{}| command to tell it not to.
\end{pset}

\end{document}
